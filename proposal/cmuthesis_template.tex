%for a more compact document, add the option openany to avoid
%starting all chapters on odd numbered pages
\documentclass[12pt]{cmuthesis}

% This is a template for a CMU thesis.  It is 18 pages without any content :-)
% The source for this is pulled from a variety of sources and people.
% Here's a partial list of people who may or may have not contributed:
%
%        bnoble   = Brian Noble
%        caruana  = Rich Caruana
%        colohan  = Chris Colohan
%        jab      = Justin Boyan
%        josullvn = Joseph O'Sullivan
%        jrs      = Jonathan Shewchuk
%        kosak    = Corey Kosak
%        mjz      = Matt Zekauskas (mattz@cs)
%        pdinda   = Peter Dinda
%        pfr      = Patrick Riley
%        dkoes = David Koes (me)

% My main contribution is putting everything into a single class files and small
% template since I prefer this to some complicated sprawling directory tree with
% makefiles.

% some useful packages
\usepackage{times}
\usepackage{fullpage}
\usepackage{graphicx}
\usepackage{amsmath}
\usepackage{cite}
\usepackage[numbers,sort]{natbib}
\usepackage[pageanchor=true,plainpages=false, pdfpagelabels, bookmarks,bookmarksnumbered,
%pdfborder=0 0 0,  %removes outlines around hyper links in online display
]{hyperref}
\usepackage{setspace}
\usepackage{subfigure}

% Approximately 1" margins, more space on binding side
%\usepackage[letterpaper,twoside,vscale=.8,hscale=.75,nomarginpar]{geometry}
%for general printing (not binding)
\usepackage[letterpaper,twoside,vscale=.8,hscale=.75,nomarginpar,hmarginratio=1:1]{geometry}

% Provides a draft mark at the top of the document. 
\draftstamp{\today}{DRAFT}

\begin {document} 
\frontmatter

%initialize page style, so contents come out right (see bot) -mjz
\pagestyle{empty}

\title{ %% {\it \huge Thesis Proposal}\\
{\bf Unsupervised Guitar String Classification for Tablature Transcription}}
\author{Jonathan Michelson}
\date{April 2017}
\Year{2017}
\trnumber{}

\committee{
Richard Stern \\
}

\support{}
\disclaimer{}

% copyright notice generated automatically from Year and author.
% permission added if \permission{} given.

\keywords{Stuff, More Stuff}

\maketitle

\begin{dedication}
For my dog
\end{dedication}

\pagestyle{plain} % for toc, was empty

%% Obviously, it's probably a good idea to break the various sections of your thesis
%% into different files and input them into this file...

\doublespacing
\begin{abstract}
Guitar tablature transcription (brief explanation) is a commonly used music notation standard that could benefit from automation. Previous work solves this task in a framework that requires prior information about the guitar. Here, an unsupervised offline solution is introduced. Inharmonicity estimates extracted from isolated guitar notes naturally segregate into semi-linear formations determined by its sourced string. The inharmonicities are modeled with a linear regression mixture, and optimized with expectation maximization. 
\end{abstract}
\singlespacing

\begin{acknowledgments}
My advisor is cool.
\end{acknowledgments}



\tableofcontents
\listoffigures
\listoftables

\mainmatter

%% Double space document for easy review:
%\renewcommand{\baselinestretch}{1.66}\normalsize

% The other requirements Catherine has:
%
%  - avoid large margins.  She wants the thesis to use fewer pages, 
%    especially if it requires colour printing.
%
%  - The thesis should be formatted for double-sided printing.  This
%    means that all chapters, acknowledgements, table of contents, etc.
%    should start on odd numbered (right facing) pages.
%
%  - You need to use the department standard tech report title page.  I
%    have tried to ensure that the title page here conforms to this
%    standard.
%
%  - Use a nice serif font, such as Times Roman.  Sans serif looks bad.
%
% Other than that, just make it look good...


\chapter{Introduction}
\doublespacing
The guitar is a widely popular musical instrument. Part of its appeal is the ease with which beginners can learn
Sound is created by pressing strings onto the neck, or fretboard, with one hand and vibrating combinations of strings over the body with the other hand. . Types: classical, acoustic, electric. Usually six strings. Open pitches, fretting them decreases vibration length and increases pitch. Different material construction, diameter, tension.

Because the pitch ranges of a guitar's adjacent strings overlap, a given musical phrase can usually be realized in a few different ways. Conventional music scores represent passages as notes and chords, and therefore leave fretboard position ambiguous; standard scores don't have a unique mapping to fretboard [see figure]. This can be problematic if one desires to capture the other important aspects of a guitar performance.

Tablature: What is it? Why is it useful?
Guitar tablature is an alternative music notation for guitarists that.... It uses ASCII characters to crudely depict the string and fret number in a musical phrase. Reasons for popularity: no musical background necessary, nice visual correspondence with actual guitar, portable and lightweight (just text). It eliminates ambiguity in the recording of a guitar passage. Largest inconveniences: typically created manually by experienced guitarist, and ground truth tabs for difficult songs are often hard to come by.

Automated tablature transcription provides a solution to tedious tab annotation, It can be used in music education for novice guitarists. It could also be used to uncover the fretboard positions of legendary guitar pieces, about whose tablature identities we can only speculate. In the literature, there are two main variants of transcription: a general version (in which goal is simply generation of a feasible tab), and transcription (in which goal is ascertainment of exact string-fret combos that produced audio).

What's the nature of the auto tab transcription problem? Existing approaches to the transcription problem focus on solutions that require prior knowledge, though in many cases this is unfeasible or undesirable. If

What I intend to do (briefly): Introduce one unsupervised solution to the transcription problem. Specifically, we take inharmonicities and plot them, cluster them with expectation maximization for linear regression mixtures, allowing us to classify strings. Then with tuning information provided by the user, we can infer the tablature.

Outline of remaining sections: Discuss existing/related work in the generation and transcription field, detailed discussion of my idea, lay out experients, what i think they'll show, introduce evaluation crtieria: how well and when done.

\singlespacing

\chapter{Previous Work}
This section introduces previous work on automated string classification or tablature transcription from strictly audio recordings. For visual, audio-visual, and instrument enhancement approaches to tablature transcription, the reader is referred to~\cite{ogrady2009, hrybyk2010}. ~\cite{abesser2012}. We intentionally omit discussion of score-generation or transcription, though much work has been done in that area and we refer the reader to [----] for more on that. \cite{barbanchoa2012,barbanchoi2012}
\cite{barbancho2009}



\chapter{Proposed Solution}
The implemented system can be decomposed into the following high-level blocks: pre-processing, inharmonicity extraction, and classification. Each block is presented in detail below.

\section{Pre-processing}
In the pre-processing stage, onset detection and pitch estimation are used to extract guitar notes from the audio files and estimate their fundamentals, respectively. The proposed system operates only on monophonic guitar recordings in isolation. For pitch detection, harmonic sum spectrum~\cite{noll1969} worked well. Rectified spectral flux with pre-processing~\cite{bello2005,dixon2006}

\section{Inharmonicity Extraction}
When a guitar string is displaced by a pluck, its rigidness contributes to the restoring force so as to skew its harmonics upward in frequency~\cite{fletcher1998}. This phenomenon, known as inharmonicity, depends on the material properties of the displaced string. The $k$th harmonic $f_k$ is related to the fundamental $f_0$ by \[f_{k} = kf_{0}\sqrt{1+\beta k^{2}}\] where \[\beta = \frac{\pi^{3}Qd^{4}}{64Tl^{2}}\] is the amount of inharmonicity, $Q$ is Young's modulus, $d$ is its diameter, $T$ is its tension, and $l$ is its vibrating length. Interestingly, this phenomenon is perceptible in the acoustic guitar in some cases (Karjalainene [cite]), but more importantly is key in the task of string discrimination. 

To motivate inharmonicity's discrimiantve power, consider equation (). If we make the simplifying assumption that, for any given string, terms $Q$, $d$, and $T$ are constant during a performance, then the inharmonicities $\beta$ of each note produced by that string are simply a function of $l^{-2}$, the inverse of the square of each notes' vibrating lengths. But note that for the guitar, length and fret number, or equivalently pitch, are interchangeable, so .......... Inharmonicities for each string are therefore restricted to a quadratic function of pitch specified by the string's unique material constants. When we plot inharmonicities versus MIDI note number, we see that the notes naturally segregate into semi-linear clusters indicative of the strings on which they were played. The semi-linearness comes from the logarithm used when transforming pitches that are harmonically exponential to a harmonically linear scale, i.e. MIDI note numbers. This is the basis for our proposed solution in the next section.

To extract the inharmonicity of a note, we follow Barbancho et. al.'s method in~\cite{barbanchoi2012}. After a note onset is detected, we take three consecutive Hanning-windowed 100-ms frames and spectrally average their DFT magnitudes. Then we search the note's magnitude spectrum for peaks close to integer multiples of the note's fundamental. We next compute a deviation vector, where each component is the amount by which these located peaks deviate from their corresponding ideal harmonics. A ---th order polynomial is fit to the deviation curve, according to: (eqn). ITERATION. The --- coefficient is now our inharmonicity estimate of this --th note. 





\section{Classification}

Once refined inharmonicity estimates of all note are obtained, we plot our note collection's pitches versus inharmonicities. (figure). We see that notes played on the same string cluster in semi-linear groups, consistent with the simplifying assumptions made above. in the polynomial funciton of frequnecy or length..... The data is amenable to a linear regression mixture model, in which one optimizes the parameters of ${k}$ linear regressions to maximize the likelihoood..... ~\cite{faria2010}

Details on EM here.....

\chapter{Preliminary Results}
The proposed solution was evaluated on the RWC dataset~\cite{goto2003}. 

\chapter{Conclusion}

%\appendix
%\include{appendix}

\backmatter

%\renewcommand{\baselinestretch}{1.0}\normalsize

% By default \bibsection is \chapter*, but we really want this to show
% up in the table of contents and pdf bookmarks.
\renewcommand{\bibsection}{\chapter{\bibname}}
%\newcommand{\bibpreamble}{This text goes between the ``Bibliography'' header and the actual list of references}
\bibliography{mybib} %your bib file
\bibliographystyle{plainnat}


\end{document}
